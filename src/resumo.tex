An�lise de ponteiros � uma das t�cnicas mais fundamentais que compiladores 
utilizam para otimizar linguagens imperativas, especialmente linguagens 
orientadas a objetos. No entanto, mesmo com toda a 
aten��o que este t�pico j� recebeu, as propostas do estado da arte atual 
presentes em compiladores ainda lidam com desafios em rela��o � precis�o e
velocidade.
Em particular, aritm�tica de ponteiros, um fator chave de linguagens como C e 
C++, ainda precisa de solu��es satisfat�rias neste campo.
Este trabalho apresenta um novo algoritmo para an�lise de ponteiros para resolver
esse problema.
O ponto chave da nossa proposta � combinar an�lise de ponteiros com an�lise 
simb�lica de largura de inteiros.
Tal combina��o nos permite desambiguar campos dentro de vetores e estruturas de 
dados, de maneira efetiva obtendo maior precis�o do que algoritmos tradicionais.
Para validar nossa t�cnica, implementamos nosso algoritmo no compilador LLVM.
Testes em uma vasta gama de benchmarks nos mostraram que podemos desambiguar 
v�rios tipos de estruturas em C que as an�lises atuais do estado da arte n�o 
conseguem lidar. Em particular, podemos desambiguar 1.35x mais compara��es 
do que as an�lises de ponteiros presentes no LLVM.
Al�m disso, nossa an�lise � r�pida: podemos lidar com um milh�o de instru��es em 
assembly em 10 segundos.

\keywords{An�lise de Ponteiros, An�lise Est�tica, Compiladores}
